% !TEX program = xelatex

\documentclass[10pt,letterpaper]{article}
\usepackage[letterpaper,margin=0.6in]{geometry}
\pagestyle{empty}
\usepackage{multicol}
\usepackage{parskip}

\usepackage{enumitem}
\setlist[itemize]{leftmargin=*,labelindent=0.1in,label=\raisebox{0.25ex}{\tiny\(\bullet\)}}

\usepackage{lmodern}
\usepackage{fontspec}
\setmainfont[Ligatures=TeX]{Noto Serif}
\usepackage{fontawesome}

\usepackage{xcolor}

\usepackage{hyperref}

\setlength{\tabcolsep}{12pt}

\newcommand*{\cvsection}[1]{
  \vspace{2mm}
  \hrule
  \vspace{-0.4em}
  \subsection*{#1}
}

\newcommand*{\cventry}[6][.25em]{
  {\bfseries #2}\hfill{\bfseries #3} \\
  \begin{tabular}{@{}l}
    {\,\,\textit{#4}}
  \end{tabular}
  \hfill\textit{#5}
  \ifx&#6&
  \else{\\
    \begin{minipage}{0.8\linewidth}
      #6
    \end{minipage}}\fi
  \par\addvspace{#1}
}

\newcommand*{\cvproject}[5][.25em]{
  {\bfseries #2}\hfill{\textit{#3}} \\
  \begin{tabular}{@{}l}
    {\,\,\textit{#4}}
  \end{tabular}
  \ifx&#5&
  \else{\\
  \begin{minipage}{0.8\linewidth}
    #5
  \end{minipage}}\fi
  \par\addvspace{#1}
}

\newcommand*{\cvpub}[6][.25em]{
  #2. ``#3.'' \textit{#4} #5 \href{#6}{#6}
}

\begin{document}

\begin{center}
  {\Huge\textbf{Kyle Roth}}

  \begin{tabular}{ c c c c c }
    \faMapMarker\enspace{\href{https://www.google.com/maps/place/Montreal,+QC,+Canada/@45.5576996,-74.0104841,10z/data=!3m1!4b1!4m5!3m4!1s0x4cc91a541c64b70d:0x654e3138211fefef!8m2!3d45.5016889!4d-73.567256}{Montréal, QC}} &
    \faGlobe\enspace{\href{https://kylrth.com}{kylrth.com}}                                                                                                                                                                   &
    \faEnvelopeO\enspace{\href{mailto:kylrth@gmail.com}{kylrth@gmail.com}}                                                                                                                                                    &
    \faGithub\enspace{\href{https://github.com/kylrth}{kylrth}}                                                                                                                                                               &
    \faMobile\enspace{\href{tel:12087049909}{+1 208 704 9909}}
  \end{tabular}
\end{center}

\vspace{2mm}

\cvsection{INTRODUCTION}

Kyle Roth is a second-year PhD candidate in the Département d'informatique et de recherche opérationelle (DIRO) at the Université de Montréal. He is advised by \href{https://www-labs.iro.umontreal.ca/~liubang/cv.pdf}{Bang Liu}. His research interests revolve around natural language processing: procedural knowledge understanding, worst-group generalization, and ethics of intelligent automation. His main research project focuses on augmenting large language models to better handle procedural knowledge.

\cvsection{EDUCATION}

{\cventry{Doctor of Philosophy (Ph.D.)}{Sep 2021 -}{Dept.~d'informatique et de rech.~opér., Université de Montréal}{Montréal, Canada}{
    \begin{itemize}
      \item \textbf{3.7 GPA}; accelerated admission in fall 2022 from M.Sc.~(\textbf{4.3 GPA})
    \end{itemize}
  }
}

{\cventry{Bachelor of Science (B.S.)}{Aug 2014 - Dec 2019}{Department of Mathematics, Brigham Young University}{Provo, USA}{
    \begin{itemize}
      \item Applied and Computational Mathematics Emphasis (\href{https://acme.byu.edu}{ACME})
      \item \textbf{3.9 GPA} (Cum Laude); minor in computer science; concentration in linguistics
    \end{itemize}
  }
}

\cvsection{WORK EXPERIENCE}

{\cventry{Cobalt Speech and Language}{Jan 2020 - Aug 2021}{speech scientist (full time)}{(remote) Provo, USA}{
    \begin{itemize}
      \item Built an online training service in Go to manage parallel training of Kaldi models on sensitive live data
      \item Implemented state-of-the-art hyperparameter selection algorithms (learning rate range test; adaptive filtering) for online training
      \item Implemented MFCC extraction in Go while avoiding allocs and array bound checks
    \end{itemize}
  }
}

{\cventry{CamachoLab, Brigham Young University}{Jan 2019 - Dec 2019}{research assistant (part time)}{Provo, USA}{
    \begin{itemize}
      \item Simulated field profiles of photonic chip components in TensorFlow using neural networks with resize convolutions
      \item Built \href{https://github.com/kylrth/slurm_gen}{SLURM\_gen}, a tool to automatically generate and manage simulated datasets in a high-performance computing environment
      \item Wrote custom resize-convolution layer to improve performance
    \end{itemize}
  }
}

{\cventry{Emergent Trading}{May 2019 - Aug 2019}{software developer (intern)}{Chicago, USA}{
    \begin{itemize}
      \item Wrote fast market analysis code in C++ to track competitors on currency markets at the Chicago Mercantile Exchange
      \item Created an interactive tool to observe currency market behavior using Bokeh
    \end{itemize}
  }
}

{\cventry{Cobalt Speech and Language}{Apr 2018 - Nov 2018}{speech scientist (intern)}{(remote) Provo, USA}{
    \begin{itemize}
      \item Improved model accuracy from 76\% to 94\% for autonomous drone recognition of air traffic control speech, using class-based (Thrax) language models
    \end{itemize}
  }
}

\cvsection{HONORS \& AWARDS}

\begin{itemize}
  \setlength\itemsep{0.25em}
  \item Université de Montréal bourse d'exemption, 3e cycle (42,076.26 CAD)\hfill\textit{Aug 2022 - Aug 2024}
  \item Université de Montréal bourse d'exemption, 2e cycle (9,789.06 CAD)\hfill\textit{Aug 2021 - Aug 2022}
  \item Brigham Young University Mathematics Department certificate of excellence\hfill\textit{Apr 2018}
  \item Brigham Young University full-tuition academic scholarship (13,500 USD)\hfill\textit{May 2017 - Dec 2019}
  \item North Idaho College mathematics student of the year\hfill\textit{May 2014}
\end{itemize}

\cvsection{PUBLICATIONS}

{\cvpub{\textbf{Kyle Roth}, Deryle Lonsdale}{Morphological Parsing and Segmentation}{BYU Journal of Undergraduate Research}{(2019): 24280.}{http://jur.byu.edu/?p=24280}}

\cvsection{RESEARCH EXPERIENCE}

{\cvproject{Mitacs Accelerate}{Aug 2022 -}{30,000 CAD. Principal research intern.}{
    \begin{itemize}
      \item \textit{Project title:} Technical and procedural knowledge extraction with question answering.
      \item \textit{Partner organization:} Thales Canada Inc.
      \item \textit{Project description:} In large organizations it's important to preserve expert knowledge with written documentation, but that documentation often contains redundant information, leaves out key details, and is difficult to search due to its open form. Our objective is to develop models that can recognize technical procedures from available documents, draw inferences about similar objects and operations, and then recognize where knowledge is incomplete so it can prompt human experts for missing information.\\
            As a part of this project, we seek to improve large language models' understanding of procedural knowledge. Retrieval-augmented generation (RAG) is widely used to improve the factuality of LLMs' outputs, but can struggle to collect information from the wide set of documents to produce accurate procedures. We are developing a method called analogy-augmented generation (AAG) to address this.
    \end{itemize}
  }
}

{\cvproject{BYU ORCA undergraduate research grant}{Jan 2018 - Dec 2018}{1,500 USD. Individual mentored research project.}{
    \begin{itemize}
      \item Project title: Morphological parsing and segmentation.
    \end{itemize}
  }
}

\cvsection{TEACHING \& SERVICE}

{\cvproject{teaching assistant}{Jan 2023 - May 2023}{Université de Montréal; IFT 6759: advanced machine learning projects}{
    \begin{itemize}
      \item Taught introductory lectures on Linux, Git, and other development tooling
    \end{itemize}
  }
}

\textbf{reviewer}
\vspace*{-1mm}
\begin{itemize}
  \setlength\itemsep{0em}
  \item 2023 - AAAI, CVPR, WWW, ACL ARR, Elsevier Pattern Recognition
  \item 2022 - Elsevier Knowledge-Based Systems
\end{itemize}

{\cvproject{volunteer}{Jul 2017 - Aug 2017}{Refugee4Refugees; Mitilini, Greece}{
  \begin{itemize}
    \item Stood night watch to spot and land refugee boats as they arrived from Turkey
    \item Taught swimming; cleaned up around Moria camp; organized donated materials
  \end{itemize}
}
}

{\cvproject{math lab tutor}{Aug 2013 - May 2014}{North Idaho College; calculus I, II, III, \& differential equations}{}{}}

\cvsection{SKILLS}

\begin{itemize}
  \item \textbf{natural languages:} native English, fluent Spanish, basic French
  \item \textbf{programming languages:} Python, Go, C++, Java, Dart, Bash, \LaTeX{}
  \item \textbf{tools:} PyTorch, TensorFlow, LangChain, SLURM, Git, Docker, Flutter, scikit-learn, AWS, SQL
        %% keywords for automated resume readers %%
        \textcolor{white}{keywords: LLMs, NumPy, Pandas, PySpark}
\end{itemize}

\end{document}
