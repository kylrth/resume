% !TEX program = xelatex

\documentclass[11pt,letterpaper]{article}
\usepackage[letterpaper,margin=0.6in]{geometry}
\pagestyle{empty}
\usepackage{multicol}
\usepackage{parskip}

\usepackage{enumitem}
\setlist[itemize]{leftmargin=*,labelindent=0.1in,itemsep=-0.5ex,label=\raisebox{0.25ex}{\tiny\(\bullet\)}}

\usepackage{lmodern}
\usepackage{fontspec}
\setmainfont[Ligatures=TeX]{Noto Serif}
\usepackage{fontawesome}

\usepackage{hyperref}

\setlength{\tabcolsep}{12pt}

\newcommand*{\cvsection}[1]{
  \vspace{2mm}
  \hrule
  \vspace*{-0.6em}
  \section*{#1}
  \vspace*{-0.4em}
}

\newcommand*{\cventry}[6][.25em]{
  {\bfseries #2}\hfill{\bfseries #3} \\
  \begin{tabular}{@{}l}
    {\,\,\itshape{} #4}
  \end{tabular}
  \hfill\textit{#5}
  \ifx&#6&
  \else{\\
    \begin{minipage}{0.77\linewidth}
      \small#6
    \end{minipage}}\fi
  \par\addvspace{#1}
}

\begin{document}

\begin{center}
  {\Huge\textbf{Kyle Roth}}

  \begin{tabular}{ c c c c c }
    \faMapMarker\enspace{\href{https://www.google.com/maps/place/Montreal,+QC,+Canada/@45.5576996,-74.0104841,10z/data=!3m1!4b1!4m5!3m4!1s0x4cc91a541c64b70d:0x654e3138211fefef!8m2!3d45.5016889!4d-73.567256}{Montréal, QC}} &
    \faGlobe\enspace{\href{https://kylrth.com}{kylrth.com}}                                                                                                                                                                   &
    \faEnvelopeO\enspace{\href{mailto:kylrth@gmail.com}{kylrth@gmail.com}}                                                                                                                                                    &
    \faGithub\enspace{\href{https://github.com/kylrth}{kylrth}}                                                                                                                                                               &
    \faMobile\enspace{\href{tel:12087049909}{+1 208 704 9909}}
  \end{tabular}
\end{center}

\vspace{2mm}

\cvsection{EDUCATION}

{\cventry{Université de Montréal}{Montréal, QC}{Ph.D., Computer Science; advised by Dr.~Bang Liu}{Sep 2021 - May 2025}{
    \begin{itemize}
      \item investigating the representation of procedural knowledge using language models
      \item \textbf{4.3 GPA} during first year as master's student
      \item switched from master's to doctoral program in August 2022
    \end{itemize}
  }
}

{\cventry{Brigham Young University}{Provo, UT}{B.S., Mathematics; Applied and Computational Mathematics Emphasis}{Aug 2014 - Dec 2019}{
    \begin{itemize}
      \item Cum Laude (\textbf{3.9 GPA}); minor in computer science; concentration in linguistics
      \item \textbf{Senior project}: scored 76\% accuracy on phoneme classification of the TIMIT corpus (research-style paper \href{https://github.com/jarednielsen/speech2phone/blob/master/results/winter_semester_report.pdf}{here})
      \item \textbf{Grant-funded research}: achieved 71\% accuracy on a Basque morphology corpus with a recent VoCRF implementation
    \end{itemize}
  }
}

\cvsection{WORK EXPERIENCE}

{\cventry{Cobalt Speech and Language}{(remote) Provo, UT}{speech scientist (full time)}{Jan 2020 - Aug 2021}{
    \begin{itemize}
      \item Built an online training service in Go to manage parallel training of Kaldi models on sensitive live data
      \item Implemented state-of-the-art hyperparameter selection algorithms (learning rate range test; adaptive filtering) for online training
      \item Implemented MFCC extraction in Go while avoiding allocs and array bound checks
    \end{itemize}
  }
}

{\cventry{Emergent Trading}{Chicago, IL}{software developer (intern)}{May 2019 - Aug 2019}{
    \begin{itemize}
      \item Wrote fast market analysis code in C++ to track competitors on currency markets at the Chicago Mercantile Exchange
      \item Designed and built an interactive tool to observe trades and prices in Brazilian currency futures using the Bokeh Python library
    \end{itemize}
  }
}

{\cventry{CamachoLab, Brigham Young University}{Provo, UT}{research assistant (part time)}{Jan 2019 - Dec 2019}{
    \begin{itemize}
      \item Simulated field profiles of photonic chip components in TensorFlow using neural networks with resize convolutions
      \item Built \href{https://github.com/kylrth/slurm_gen}{SLURM\_gen}, a tool to automatically generate and manage simulated datasets in a high-performance computing environment
      \item Wrote custom resize-convolution layer to improve performance
    \end{itemize}
  }
}

{\cventry{Cobalt Speech and Language}{(remote) Provo, UT}{speech scientist (intern)}{Apr 2018 - Nov 2018}{
    \begin{itemize}
      \item Improved model accuracy from 76\% to 94\% for autonomous drone recognition of air traffic control speech, using class-based (Thrax) language models
    \end{itemize}
  }
}

\cvsection{SKILLS}

\begin{minipage}{\linewidth}
  \small\begin{multicols}{2}
    \begin{itemize}
      \setlength\itemsep{0ex}
      \item \textbf{languages:} Python, Go, C++, Java, Dart, Bash, \LaTeX{}
      \item \textbf{tools:} PyTorch, TensorFlow, SLURM, Kaldi, git, scikit-learn, NumPy, Pandas, AWS, SQL, PySpark
      \item \textbf{natural languages:} native English, fluent Spanish, basic French
    \end{itemize}
  \end{multicols}
\end{minipage}

\end{document}
